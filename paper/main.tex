% Two Economies? Stock Markets, Job Postings, and AI Exposure in Sweden
% Economics Letters — max 2,000 words (incl. tables, figures, references)
% LaTeX template: elsarticle (Elsevier)

\documentclass[preprint,12pt]{elsarticle}

\usepackage[utf8]{inputenc}
\usepackage[T1]{fontenc}
\usepackage{amsmath,amssymb}
\usepackage{graphicx}
\usepackage{booktabs}
\usepackage{hyperref}
\usepackage{natbib}
\usepackage{setspace}

% Path to figures
\graphicspath{{../figures/}}

\journal{Economics Letters}

\begin{document}

\begin{frontmatter}

\title{Two Economies? Stock Markets, Job Postings, and AI Exposure in Sweden}

\author[oru]{Magnus Lodefalk\corref{cor}}
\ead{magnus.lodefalk@oru.se}
\author[oru]{Erik Engberg}
\author[au]{Michael Koch}
\author[oru]{Lydia L\"othman}

\cortext[cor]{Corresponding author}
\address[oru]{\"Orebro University School of Business, SE-701 82 \"Orebro, Sweden}
\address[au]{Department of Economics and Business Economics, Aarhus University, Denmark}

\begin{abstract}
% TARGET: 95 words (max 100)
A ``scary chart'' shows US stock prices rising while job postings fall --- a pattern attributed to AI displacement. We test this using 4.6 million Swedish job ads (2020--2026) from Platsbanken matched to a generative AI exposure index. A difference-in-differences design exploits the Riksbanken rate hike (April 2022) preceding ChatGPT (November 2022) by seven months. Postings declined broadly across all AI exposure quartiles beginning with monetary tightening, with no significant additional decline after ChatGPT in the baseline specification. The posting gap primarily reflects macroeconomic tightening. All data and code are publicly available.
\end{abstract}

\begin{keyword}
Job postings \sep Artificial intelligence \sep Monetary policy \sep Labour demand \sep Sweden
\end{keyword}

\end{frontmatter}

% ══════════════════════════════════════════════════════════════════════════════
\section{Introduction}
% TARGET: ~400 words
% ══════════════════════════════════════════════════════════════════════════════

A widely circulated chart shows US stock prices rising roughly 70\% since 2020 while job postings on Indeed fell approximately 30\% --- suggesting ``two economies,'' one thriving in financial markets and one contracting in the labour market, with AI as the suspected wedge \citep{thompson2025scary}. \citet{brynjolfsson2025canaries} provided the first systematic evidence, finding that US entry-level workers in AI-automatable occupations experienced employment declines since late 2022 --- ``canaries in the coal mine'' signalling broader displacement.

Two studies have tested this outside the US. \citet{kauhanen2025canaries} found no employment decline among young Finnish workers in high-AI-exposure occupations using population register data; instead, wages rose more in these occupations. \citet{kauhanen2025youth} extended the analysis and confirmed no youth labour market effects in Finland through 2024.

We make three contributions. First, we study \emph{job postings} rather than employment. While \citet{brynjolfsson2025canaries} and the Finnish replications test whether young workers in AI-exposed occupations lose employment, we test a related upstream margin: whether firms reduce \emph{hiring} in those occupations. Posting counts capture the flow of labour demand and are the source of the ``scary chart'' itself, though AI may also reshape the task content of postings without reducing their number \citep{autor2024applying}. Our data do not contain worker age, so we cannot replicate the age-specific ``canaries'' test; instead, we decompose the aggregate posting decline by AI exposure. Second, we use individual-level microdata from Platsbanken, the Swedish Public Employment Service's job bank, comprising over 4.6 million ads (2020--2026) matched to the DAIOE generative AI exposure index at the SSYK 4-digit occupation level \citep{lodefalk2024daioe}. Third, we exploit a natural timing test: Sweden's central bank began raising rates in April 2022 --- seven months before ChatGPT launched in November 2022. The rate hike coincided with Russia's invasion of Ukraine and the ensuing European energy crisis, creating a constellation of macroeconomic headwinds distinct from AI. If AI drives the posting decline in exposed occupations, the differential should emerge or widen after ChatGPT. If macroeconomic tightening is the primary driver, the decline should begin with the rate hike.

Our difference-in-differences estimates show that postings declined broadly across all AI exposure quartiles starting in mid-2022, with no statistically significant additional decline in high-exposure occupations after ChatGPT in the baseline specification, though some alternative exposure measures yield suggestive effects.


% ══════════════════════════════════════════════════════════════════════════════
\section{Data and method}
% TARGET: ~450 words
% ══════════════════════════════════════════════════════════════════════════════

\subsection{Platsbanken microdata}

We use the full population of job advertisements published on Platsbanken from January 2020 through February 2026, downloaded as historical JSONL files from JobTech Development.\footnote{Available at \url{https://data.jobtechdev.se/annonser/historiska/}. CC0 licence.} Each ad contains a publication date, SSYK 2012 four-digit occupation code, number of vacancies, municipality, and employer. After removing ads without valid SSYK codes and deduplicating on ad identifier, we retain 4.6 million ads across 400 occupations, which we aggregate to occupation $\times$ year-month cells (26,672 observations). Not all Swedish vacancies appear on Platsbanken; white-collar professional roles are increasingly posted on LinkedIn and other platforms. Since these are disproportionately high-AI-exposure occupations, platform migration would reduce measured postings in Q4 regardless of AI, biasing us \emph{toward} finding a negative AI effect. Our null result for $\hat\beta_2$ is therefore conservative.

\subsection{AI exposure}

We measure generative AI exposure using the DAIOE index \citep{lodefalk2024daioe}, which maps AI benchmark capabilities to Swedish occupations at the SSYK 4-digit level. We use the 2023 cross-section of the percentile ranking for generative AI (\texttt{pctl\_rank\_genai}) and classify occupations into quartiles, where Q4 is the most exposed. The match rate between Platsbanken SSYK codes and DAIOE is 92\% at the occupation level (369 of 400 codes), covering 96\% of all ads.

\subsection{Identification strategy}

We estimate:
\begin{equation}\label{eq:did}
\ln(\text{postings}_{it}) = \alpha_i + \gamma_t + \beta_1 \cdot \text{PostRB}_t \cdot \text{High}_i + \beta_2 \cdot \text{PostGPT}_t \cdot \text{High}_i + \varepsilon_{it}
\end{equation}
where $i$ indexes SSYK4 occupations and $t$ indexes year-months; $\alpha_i$ and $\gamma_t$ are occupation and time fixed effects; $\text{PostRB}_t = 1$ if $t \geq$ April 2022 (Riksbanken's first rate hike); $\text{PostGPT}_t = 1$ if $t \geq$ December 2022; $\text{High}_i = 1$ if occupation $i$ is in the top quartile of genAI exposure. Standard errors are clustered at the occupation level.

The key test: if AI drives the differential decline, $\hat\beta_2$ should be significantly negative. If monetary policy drives it, $\hat\beta_1$ should be significant with $\hat\beta_2 \approx 0$. We also estimate specifications with occupation-specific time trends.

\subsection{Auxiliary data}

OMXS30 daily prices from Yahoo Finance, resampled to monthly averages and indexed to 100 at February 2020.


% ══════════════════════════════════════════════════════════════════════════════
\section{Results}
% TARGET: ~550 words
% ══════════════════════════════════════════════════════════════════════════════

\subsection{The Swedish scary chart}

Figure~\ref{fig:scary} shows the Swedish version of the ``scary chart.'' OMXS30 rose to over 175\% of its February 2020 level by early 2026, while job postings across all AI exposure quartiles declined following the Riksbanken rate hike in April 2022. The decline is broad-based: Q4 (highest genAI exposure) and Q1 (lowest) follow similar trajectories, though Q4 declines somewhat more. The divergence between stock prices and postings opens seven months before ChatGPT, coinciding with the start of monetary tightening. The pattern is identical when using the OMXSPI All-Share index instead of the OMXS30 (see Online Appendix).

\begin{figure}[htbp]
\centering
\includegraphics[width=\textwidth]{fig1_scary_chart.png}
\caption{Stock market vs job postings by AI exposure quartile, Sweden 2020--2026. OMXS30 index (left axis) and Platsbanken posting index by genAI exposure quartile (right axis), both indexed to 100 at February 2020. Vertical lines mark Riksbanken's first rate hike (April 2022) and ChatGPT launch (November 2022).}
\label{fig:scary}
\end{figure}

\subsection{Exposure gap}

Figure~\ref{fig:gap} plots the Q4 minus Q1 posting gap over time. The gap fluctuates around zero throughout the sample period, including after the ChatGPT launch. There is no visible widening of the gap following November 2022, as would be expected if AI were differentially reducing postings in exposed occupations.

\begin{figure}[htbp]
\centering
\includegraphics[width=\textwidth]{fig2_exposure_gap.png}
\caption{Posting index gap between Q4 (highest genAI exposure) and Q1 (lowest), with 3-month moving average and 95\% confidence band. No systematic widening after ChatGPT.}
\label{fig:gap}
\end{figure}

\subsection{Regression results}

Table~\ref{tab:did} reports the DiD estimates. In specification (1), the Riksbanken interaction alone is strongly significant ($\hat\beta_1 = -0.178$, $p < 0.01$), indicating that high-exposure occupations experienced a differential posting decline after the rate hike. In specification (2), adding the ChatGPT interaction yields $\hat\beta_2 = -0.062$ ($p = 0.11$) --- negative but statistically insignificant --- while the monetary policy interaction remains significant ($\hat\beta_1 = -0.127$, $p < 0.01$). Specification (3) adds occupation-specific time trends; both interactions lose significance ($\hat\beta_1 = -0.068$, $\hat\beta_2 = 0.018$), suggesting limited power to separate closely spaced shocks. Our preferred specification (4) replaces common month fixed effects with SSYK 1-digit occupation group $\times$ month fixed effects, comparing high- and low-exposure occupations within the same broad group. Both coefficients become insignificant, indicating that the differential decline in specification (2) reflects different macro sensitivity across occupation groups --- knowledge-intensive services respond more to interest rates than manual occupations --- rather than within-group AI exposure. The minimum detectable effect at 80\% power is approximately 10 log points, implying that our design cannot rule out economically meaningful effects below this threshold.

\begin{table}[htbp]
\centering
\caption{Difference-in-differences: postings by genAI exposure}
\label{tab:did}
\begin{tabular}{lcccc}
\hline\hline
 & (1) & (2) & (3) & (4) \\
 & Monetary & + ChatGPT & + Trends & Group$\times$Time \\
\hline
Post-Riksbank $\times$ High & -0.178*** & -0.127*** & -0.068 & -0.039 \\
 & (0.041) & (0.039) & (0.050) & (0.049) \\
Post-ChatGPT $\times$ High &  & -0.062 & 0.018 & -0.032 \\
 &  & (0.038) & (0.040) & (0.048) \\
Time trend $\times$ High &  &  & -0.003** &  \\
 &  &  & (0.001) &  \\
\hline
Occupation FE & Yes & Yes & Yes & Yes \\
Month FE & Yes & Yes & Yes & \\
Occ.\ group $\times$ month FE & & & & Yes \\
Dep.\ var.\ mean & \multicolumn{4}{c}{3.77} \\
Occupations & \multicolumn{4}{c}{369} \\
Observations & \multicolumn{4}{c}{26,672} \\
\hline\hline
\multicolumn{5}{p{0.95\textwidth}}{\footnotesize \textit{Notes:} Dependent variable: $\ln(\text{postings}_{it})$. High = top quartile of DAIOE genAI percentile ranking. Post-Riksbank $=$ April 2022 onward. Post-ChatGPT $=$ December 2022 onward. Column (4) replaces common month FE with SSYK 1-digit occupation group $\times$ month FE. Standard errors clustered at occupation level in parentheses. $^{***}p<0.01$, $^{**}p<0.05$, $^{*}p<0.10$.} \\
\end{tabular}
\end{table}

Six of eight alternative specifications confirm the insignificance of $\hat\beta_2$. However, the all-apps AI exposure measure and tercile classification yield significant ChatGPT coefficients ($p = 0.018$ and $p = 0.026$ respectively), and $\hat\beta_2$ is negative in seven of eight specifications. Adding quadratic occupation-specific trends does not alter the conclusion: the quadratic term is insignificant ($p = 0.56$). An event study rejects parallel pre-trends even at quarterly frequency, but the pre-period differentials reflect that knowledge-intensive occupations (high AI exposure) are more cyclically sensitive --- an economically intuitive pattern distinct from AI. A sensitivity analysis following \citet{rambachan2023more} finds that the average post-ChatGPT event-study effect ($\hat\theta = -0.169$) loses significance at a breakdown value of $\bar{M} = 0.25$, confirming the fragility of the AI channel (see Online Appendix).


% ══════════════════════════════════════════════════════════════════════════════
\section{Discussion}
% TARGET: ~200 words
% ══════════════════════════════════════════════════════════════════════════════

Our findings complement \citet{kauhanen2025canaries} and \citet{kauhanen2025youth}, who found no AI-driven employment effects in Finland. We cannot cleanly distinguish ``no AI effect'' from ``a modest AI effect our design lacks power to detect'': $\hat\beta_2$ is negative in seven of eight alternative specifications, significant in two, and the minimum detectable effect (10 log points) exceeds the point estimate. A supplementary analysis using SCB register employment data by age group finds no canaries effect (see Online Appendix). Together, these Nordic studies suggest that the US ``canaries'' finding may partly reflect country-specific labour market institutions; strong collective bargaining, generous social insurance, and active labour market policies in Sweden and Finland may buffer AI-induced displacement.

Two caveats merit emphasis. First, specification (4) shows that conditioning on occupation-group $\times$ month fixed effects eliminates both interactions, suggesting that the differential decline reflects different macro sensitivity across broad occupation groups rather than a within-group AI effect. Second, the $\text{Post-RB}$ dummy captures not only the rate hike but also the energy crisis and other macroeconomic headwinds of early 2022; our design identifies ``macroeconomic tightening versus AI,'' not ``monetary policy versus AI'' narrowly.

Our sample ends in early 2026. As AI adoption accelerates, future data may reveal displacement effects not yet detectable. Our publicly available pipeline enables ongoing monitoring.

% ══════════════════════════════════════════════════════════════════════════════
% AI disclosure (required by Economics Letters / Elsevier)
% ══════════════════════════════════════════════════════════════════════════════

\section*{AI disclosure}
During the preparation of this work, the authors used Claude (Anthropic) for code development, data processing, and editorial suggestions. The authors reviewed and edited all content and take full responsibility for the publication.

\section*{Data availability}
All data and replication code are available at \url{https://github.com/Magnus-L/canaries-sweden}.

\section*{Acknowledgements}
Financial support from the Marianne and Marcus Wallenberg Foundation (WASP-HS, project AISCAF) is gratefully acknowledged.

% ══════════════════════════════════════════════════════════════════════════════
\bibliographystyle{elsarticle-harv}
\bibliography{references}

\end{document}
