% Two Economies? Stock Markets, Job Postings, and AI Exposure in Sweden
% Economics Letters — max 2,000 words (incl. tables, figures, references)
% LaTeX template: elsarticle (Elsevier)

\documentclass[preprint,12pt,authoryear]{elsarticle}

\usepackage[utf8]{inputenc}
\usepackage[T1]{fontenc}
\usepackage{amsmath,amssymb}
\usepackage{graphicx}
\usepackage{booktabs}
\usepackage{hyperref}
\usepackage{natbib}
\usepackage{setspace}

% Path to figures
\graphicspath{{../figures/}}

\journal{Economics Letters}

\begin{document}

\begin{frontmatter}

\title{Two Economies? Stock Markets, Job Postings, and AI Exposure in Sweden}

\author[oru,ratio]{Magnus Lodefalk\corref{cor}}
\ead{magnus.lodefalk@oru.se}
\author[oru,ratio]{Erik Engberg}
\author[au]{Michael Koch}
\author[oru]{Lydia L\"othman}

\cortext[cor]{Corresponding author}
\address[oru]{\"Orebro University School of Business, SE-701 82 \"Orebro, Sweden}
\address[ratio]{RATIO Institute, Stockholm, Sweden}
\address[au]{Department of Economics and Business Economics, Aarhus University, Denmark}

\begin{abstract}
% TARGET: ~100 words (max 100)
A ``scary chart'' shows US stock prices rising while job postings fall, widely attributed to AI displacement. We test this using 4.6 million Swedish job ads (2020--2026) matched to a generative AI exposure index. A difference-in-differences design leverages the Riksbank rate hike (April 2022) preceding ChatGPT (November 2022) by seven months. Postings declined across all AI exposure quartiles beginning with monetary tightening, with no statistically significant additional decline after ChatGPT. Yet monthly population-register data suggest a canaries effect: young workers in AI-exposed occupations experienced disproportionate employment declines. The posting gap reflects macroeconomic tightening; AI operates on the composition of employment.
\end{abstract}

\begin{keyword}
Job postings \sep Artificial intelligence \sep Monetary policy \sep Labour demand \sep Sweden
\end{keyword}

\end{frontmatter}

% ══════════════════════════════════════════════════════════════════════════════
\section{Introduction}
% TARGET: ~400 words
% ══════════════════════════════════════════════════════════════════════════════

A widely circulated chart shows US stock prices rising about 70\% since 2020 while job postings on Indeed fell 30\%, suggesting ``two economies,'' one thriving in financial markets and one contracting in the labour market, with AI as the conjectured driver \citep{thompson2025scary}. \citet{brynjolfsson2025canaries} provided the first systematic evidence: employment among US entry-level workers in AI-exposed occupations declined from late 2022, ``canaries in the coal mine'' signalling broader displacement.

Two studies have tested the canaries hypothesis outside the US. \citet{kauhanen2025canaries} found no employment decline among young Finnish workers in high-AI-exposure occupations using population register data; instead, wages rose more in these occupations. \citet{kauhanen2025youth} extended the analysis and confirmed no youth labour market effects in Finland through 2024.

We test \emph{both} the posting margin and the employment margin. Using 4.6 million job ads from Platsbanken (2020--2026) matched to the DAIOE index \citep{engberg2024daioe}, we decompose the aggregate posting decline by AI exposure and complement this with monthly population-register employment data by age group. We exploit a natural timing test: the Riksbank began raising rates in April 2022, seven months before ChatGPT launched in November 2022. The rate hike coincided with Russia's invasion of Ukraine and the European energy crisis, creating macroeconomic shocks unrelated to AI. If AI drives the differential posting decline, it should emerge after ChatGPT; if monetary tightening is the primary driver, it should begin with the rate hike. We also provide the first non-US evidence of a canaries effect in employment, contrasting with the Finnish null \citep{kauhanen2025canaries,kauhanen2025youth}.

Our difference-in-differences estimates show that postings declined across all AI exposure quartiles starting in mid-2022, with no significant additional decline in high-exposure occupations after ChatGPT. Yet monthly population-register data suggest a canaries effect: young workers in AI-exposed occupations experienced disproportionate declines, consistent with \citet{brynjolfsson2025canaries}. The posting gap reflects macroeconomic tightening; AI reshapes the composition of hiring within occupations.


% ══════════════════════════════════════════════════════════════════════════════
\section{Data and method}
% TARGET: ~450 words
% ══════════════════════════════════════════════════════════════════════════════

\subsection{Platsbanken microdata}

We use the full population of job advertisements published on Platsbanken from January 2020 through February 2026, downloaded as historical JSONL files from JobTech Development.\footnote{Available at \url{https://data.jobtechdev.se/annonser/historiska/}. CC0 licence.} Each ad contains a publication date, SSYK 2012 four-digit occupation code, number of vacancies, municipality, and employer. After removing ads without valid SSYK codes and deduplicating on ad identifier, we retain 4.6 million ads across 400 occupations, which we aggregate to occupation $\times$ year-month cells (26,672 observations). Not all Swedish vacancies appear on Platsbanken; white-collar professional roles are increasingly posted on LinkedIn and other platforms. Since these are disproportionately high-AI-exposure occupations, platform migration would reduce measured postings in Q4 regardless of AI, biasing us \emph{toward} finding a negative AI effect. Our null result for $\hat\beta_2$ is therefore conservative.

\subsection{AI exposure}

We measure generative AI exposure using the DAIOE index \citep{engberg2024daioe}, which maps AI benchmark capabilities to Swedish occupations at the SSYK 4-digit level. We use the 2023 cross-section of the percentile ranking for generative AI (\texttt{pctl\_rank\_genai}) and classify occupations into quartiles, where Q4 is the most exposed. The match rate between Platsbanken SSYK codes and DAIOE is 92\% at the occupation level (369 of 400 codes), covering 96\% of all ads.

\subsection{Identification strategy}

We estimate:
\begin{equation}\label{eq:did}
\ln(\text{postings}_{it}) = \alpha_i + \gamma_t + \beta_1 \cdot \text{PostRB}_t \cdot \text{High}_i + \beta_2 \cdot \text{PostGPT}_t \cdot \text{High}_i + \varepsilon_{it}
\end{equation}
where $i$ indexes SSYK4 occupations and $t$ indexes year-months; $\alpha_i$ and $\gamma_t$ are occupation and time fixed effects; $\text{PostRB}_t = 1$ if $t \geq$ April 2022 (the Riksbank's first rate hike); $\text{PostGPT}_t = 1$ if $t \geq$ December 2022 (the first full month after ChatGPT's November~30 launch); $\text{High}_i = 1$ if occupation $i$ is in the top quartile of genAI exposure. Standard errors are clustered at the occupation level.

The key test: if AI drives the differential decline, $\hat\beta_2$ should be significantly negative. If monetary policy drives it, $\hat\beta_1$ should be significant with $\hat\beta_2 \approx 0$. We also estimate specifications with occupation-specific time trends.

\subsection{Auxiliary data}

OMXS30 daily prices from Yahoo Finance, resampled to monthly averages and indexed to 100 at February 2020. For the employment analysis, we use monthly employer declaration (AGI) register data from SCB's secure research environment (MONA). These data cover the population of employed individuals, with each worker matched to an SSYK~4-digit occupation, age, and gender. AI exposure is assigned from the DAIOE quartile of the worker's most recent occupation (2023 register year).


% ══════════════════════════════════════════════════════════════════════════════
\section{Results}
% TARGET: ~550 words
% ══════════════════════════════════════════════════════════════════════════════

\subsection{The Swedish scary chart}

Figure~\ref{fig:scary} shows the Swedish version of the ``scary chart.'' OMXS30 rose to over 175\% of its February 2020 level by early 2026, while job postings across all AI exposure quartiles declined following the Riksbank rate hike in April 2022. The decline is broad-based: Q4 (highest genAI exposure) and Q1 (lowest) follow similar trajectories, though Q4 declines somewhat more. The divergence between stock prices and postings opens seven months before ChatGPT, coinciding with the start of monetary tightening. The pattern is identical when using the OMXSPI All-Share index instead of the OMXS30 (see Online Appendix).

\begin{figure}[htbp]
\centering
\includegraphics[width=\textwidth]{fig1_scary_chart.png}
\caption{Stock market vs job postings by AI exposure quartile, Sweden 2020--2026. OMXS30 index (left axis) and Platsbanken posting index by genAI exposure quartile (right axis), both indexed to 100 at February 2020. Vertical lines mark the Riksbank's first rate hike (April 2022) and ChatGPT launch (November 2022).}
\label{fig:scary}
\end{figure}

\subsection{Postings: no differential AI effect}

A difference-in-differences design (Equation~\ref{eq:did}) yields no statistically significant additional decline in high-AI-exposure occupations after ChatGPT ($\hat\beta_2 = -0.062$, $p = 0.11$), while the Riksbank rate-hike interaction is precisely estimated ($\hat\beta_1 = -0.127$, $p < 0.01$). Replacing common time fixed effects with occupation-group $\times$ month fixed effects eliminates both interactions, indicating that the differential reflects different macro sensitivity across occupation groups rather than AI exposure. $\hat\beta_2$ is insignificant in six of eight robustness specifications but negative in seven, a sign pattern consistent with a small effect that our data lack power to detect. A sensitivity analysis following \citet{rambachan2023more} yields a breakdown value of $\bar{M} = 0.25$, confirming the fragility of the AI channel.

One exception emerges when splitting by teleworkability \citep{dingel2020many}: the ChatGPT coefficient is essentially zero in teleworkable occupations ($\hat\beta_2 = -0.005$, $p = 0.92$) but strongly significant in non-teleworkable ones ($\hat\beta_2 = -0.233$, $p < 0.01$), suggesting that AI posting displacement, where it occurs, concentrates in occupations where remote work cannot cushion the impact. Full regression tables, event studies, and robustness checks are in the Online Appendix.\footnote{All regression results independently verified using Stata~18.5 \texttt{reghdfe}; coefficients match to three decimal places.}

\subsection{Employment: canaries in the coal mine}

Figure~\ref{fig:canaries} tells a different story. Using monthly AGI register data, we plot employment trajectories for four groups defined by age (16--24 vs 25+) and AI exposure (top quartile vs rest). Older workers are unaffected regardless of AI exposure, with stable trajectories throughout 2019--2025. Young workers in both exposure groups decline after 2022, but those in high-AI-exposure occupations experience a dramatically steeper fall, from indexed peaks around 120--130 to approximately 50 by late 2025, an apparent near-halving of employment. The divergence between young workers in high and low AI exposure emerges after the ChatGPT launch and widens continuously through 2025. This pattern is consistent with the canaries effect documented by \citet{brynjolfsson2025canaries} in the US. These patterns are descriptive; formal causal testing of the employment margin awaits individual-level panel analysis. Spotlight analyses of specific occupations (software developers, SSYK~2512, and customer service agents) show similar age-graded patterns, with the youngest cohorts (22--25) most affected (see Online Appendix).\footnote{Swedish bachelor's graduates are at minimum 22, so the 16--24 bracket captures pre-graduation and very-early-career workers. The occupation is assigned from the 2023 register; a graduate entering a high-AI occupation in 2024 may be coded to their 2023 student job (e.g., shop assistant), attenuating the measured differential. Our finding is therefore conservative.}

\begin{figure}[htbp]
\centering
\includegraphics[width=\textwidth]{figA8c_mona_canaries_economy.png}
\caption{Monthly employment by age group and AI exposure, Sweden 2019--2025. AGI register data indexed to base month = 100. Young (16--24) workers in high-AI-exposure occupations (top quartile, DAIOE) experience a sharp divergence after ChatGPT launch, while older workers are unaffected. Vertical lines mark the Riksbank's first rate hike (April 2022) and ChatGPT launch (November 2022).}
\label{fig:canaries}
\end{figure}


% ══════════════════════════════════════════════════════════════════════════════
\section{Discussion}
% TARGET: ~200 words
% ══════════════════════════════════════════════════════════════════════════════

The contrast between the posting null and the employment canaries is the paper's key finding. Our posting results complement \citet{kauhanen2025canaries} and \citet{kauhanen2025youth}, who found no AI-driven employment effects among Finnish youth. In Sweden, the aggregate posting decline masks a compositional shift: firms post vacancies at similar rates across AI exposure levels, yet young workers in AI-exposed occupations are disproportionately losing employment, consistent with \citet{brynjolfsson2025canaries}.

Three caveats apply. First, the posting null may partly reflect limited statistical power; the minimum detectable effect is approximately 10 log points, and the point estimate is negative in seven of eight specifications. A supplementary analysis finds no significant correlation between AI exposure and monetary policy sensitivity (see Online Appendix). Second, $\text{Post-RB}$ captures the full constellation of early-2022 macroeconomic shocks, not monetary policy alone; all advanced economies experienced synchronised tightening before ChatGPT, so this identification challenge is not specific to Sweden. Third, the employment evidence is descriptive and cannot rule out that the age-specific pattern reflects cyclical sensitivity of entry-level hiring rather than AI displacement. Nonetheless, the composition-not-level distinction matters: aggregate posting declines should not be attributed to AI without examining who is hired.

% ══════════════════════════════════════════════════════════════════════════════
% AI disclosure (required by Economics Letters / Elsevier)
% ══════════════════════════════════════════════════════════════════════════════

\section*{AI disclosure}
During the preparation of this work, the authors used Claude (Anthropic) for code development, data processing, and editorial suggestions. The authors reviewed and edited all content and take full responsibility for the publication.

\section*{Data availability}
Platsbanken job advertisements are publicly available (CC0 licence). The employment analysis uses confidential administrative microdata from Statistics Sweden (SCB), accessed via the MONA platform; details on data access are in the Online Appendix. All replication code is available at \url{https://github.com/Magnus-L/canaries-sweden}.

\section*{Acknowledgements}
Lodefalk acknowledges financial support from the Torsten Söderberg Foundation (grants E46/21, ET3/23), and Lodefalk and Löthman from WASP-HS (grant 805). Ethical approvals: 2021-05040, 2022-03330-02, 2024-01714-0, 2025-04205-02.

% ══════════════════════════════════════════════════════════════════════════════
\bibliographystyle{elsarticle-harv}
\bibliography{references}

\end{document}
