% Online Appendix for "Two Economies?"
% This supplementary material does not count toward the 2,000-word limit.

\documentclass[preprint,12pt]{elsarticle}

\usepackage[utf8]{inputenc}
\usepackage[T1]{fontenc}
\usepackage{amsmath,amssymb}
\usepackage{graphicx}
\usepackage{booktabs}
\usepackage{hyperref}

\graphicspath{{../figures/}}

\begin{document}

\begin{frontmatter}
\title{Online Appendix: Two Economies? Stock Markets, Job Postings, and AI Exposure in Sweden}
\author{Lodefalk, Engberg, and L\"othman}
\end{frontmatter}

\setcounter{section}{0}
\renewcommand{\thesection}{A\arabic{section}}
\renewcommand{\thetable}{A\arabic{table}}
\renewcommand{\thefigure}{A\arabic{figure}}


% ══════════════════════════════════════════════════════════════════════════════
\section{Sweden vs United States comparison}
% ══════════════════════════════════════════════════════════════════════════════

Figure~\ref{fig:svdus} shows side-by-side comparisons of the stock market versus job postings divergence in the United States and Sweden. The US panel uses the S\&P~500 and Indeed Hiring Lab aggregate job postings index; the Swedish panel uses OMXS30 and Platsbanken microdata. Both countries exhibit a similar qualitative pattern of rising stock markets and falling postings since mid-2022.

\begin{figure}[htbp]
\centering
\includegraphics[width=\textwidth]{figA1_sweden_vs_us.png}
\caption{Stock markets vs job postings: United States (left) and Sweden (right), both indexed to 100 at February 2020.}
\label{fig:svdus}
\end{figure}


% ══════════════════════════════════════════════════════════════════════════════
\section{Individual quartile trends}
% ══════════════════════════════════════════════════════════════════════════════

Figure~\ref{fig:quartiles} shows each AI exposure quartile's posting trajectory individually. All four quartiles exhibit similar cyclical patterns, peaking in mid-2022 and declining through 2023--2024. The parallelism of these trends is consistent with a common macroeconomic driver rather than AI-specific displacement.

\begin{figure}[htbp]
\centering
\includegraphics[width=\textwidth]{figA2_quartile_panels.png}
\caption{Job posting index by genAI exposure quartile (Feb 2020 = 100), individual panels.}
\label{fig:quartiles}
\end{figure}


% ══════════════════════════════════════════════════════════════════════════════
\section{Top and bottom occupations}
% ══════════════════════════════════════════════════════════════════════════════

Table~\ref{tab:topbottom} lists the ten most and ten least genAI-exposed occupations according to the DAIOE index.

\begin{table}[htbp]
\centering
\caption{Most and least genAI-exposed occupations (DAIOE)}
\label{tab:topbottom}
\begin{tabular}{clc}
\hline\hline
SSYK & Occupation & GenAI pctl \\
\hline
\multicolumn{3}{l}{\textit{Most exposed (top 10)}} \\
2641 & Authors and related writers & 99.9 \\
2122 & Statisticians & 99.7 \\
2121 & Mathematicians and actuaries & 99.4 \\
2415 & Economists & 99.2 \\
2512 & Software and systems developers & 98.9 \\
2145 & Chemical engineers & 98.7 \\
2111 & Physicists and astronomers & 98.5 \\
2414 & Securities traders and fund managers & 98.2 \\
2513 & Game and digital media developers & 98.0 \\
2112 & Meteorologists & 97.8 \\
\hline
\multicolumn{3}{l}{\textit{Least exposed (bottom 10)}} \\
9120 & Vehicle, window and related cleaners & 2.2 \\
2653 & Dancers and choreographers & 2.0 \\
8350 & Ships' deck crew and related workers & 1.8 \\
7113 & Concrete workers & 1.5 \\
8341 & Agricultural and forestry machinery operators & 1.3 \\
9310 & Construction labourers & 1.1 \\
8342 & Earth-moving machinery operators & 0.8 \\
8111 & Miners and quarry workers & 0.6 \\
7121 & Roofers & 0.3 \\
3421 & Professional athletes & 0.1 \\
\hline\hline
\end{tabular}
\end{table}


% ══════════════════════════════════════════════════════════════════════════════
\section{Robustness checks}
% ══════════════════════════════════════════════════════════════════════════════

Table~\ref{tab:robustness} reports results from alternative specifications: (i)~using the all-apps AI exposure measure instead of genAI; (ii)~using vacancy-weighted postings; (iii)~excluding pandemic months (January--June 2020); (iv)~using tercile classification instead of quartiles; (v)~excluding IT/tech occupations (SSYK~25xx), following \citet{brynjolfsson2025canaries}; (vi)~restricting to a balanced panel; (vii)~using language-modelling task exposure from DAIOE. The main finding --- no significant additional decline in high-exposure postings after ChatGPT --- is robust across all specifications.

\begin{table}[htbp]
\centering
\caption{Robustness checks}
\label{tab:robustness}
\begin{tabular}{lcccc}
\hline\hline
Specification & $\hat\beta_1$ (Riksbank) & $\hat\beta_2$ (ChatGPT) & $N$ & Occ. \\
\hline
Baseline (genAI Q4) & -0.1271*** & -0.0615 & 26,672 & 369 \\
 & (0.0388) & (0.0380) & & \\
All-apps measure & -0.0921** & -0.0911** & 26,672 & 369 \\
 & (0.0386) & (0.0385) & & \\
Vacancy-weighted & -0.1038** & -0.0612 & 26,672 & 369 \\
 & (0.0443) & (0.0437) & & \\
Excl. pandemic & -0.1159*** & -0.0617 & 24,479 & 369 \\
 & (0.0356) & (0.0380) & & \\
Terciles & -0.0936*** & -0.0808** & 26,672 & 369 \\
 & (0.0360) & (0.0362) & & \\
Excl. IT/tech & -0.1291*** & -0.0662* & 26,148 & 362 \\
 & (0.0406) & (0.0400) & & \\
Balanced panel & -0.1130*** & -0.0325 & 22,176 & 308 \\
 & (0.0377) & (0.0345) & & \\
Language model & -0.1154*** & -0.0564 & 26,672 & 369 \\
 & (0.0386) & (0.0387) & & \\
Quadratic trends & -0.0884*** & 0.0193 & 26,672 & 369 \\
 & (0.0323) & (0.0402) & & \\
\hline\hline
\multicolumn{5}{p{0.95\textwidth}}{\footnotesize \textit{Notes:} All specifications include occupation and month fixed effects. Standard errors (in parentheses) clustered at occupation level. $^{***}p<0.01$, $^{**}p<0.05$, $^{*}p<0.10$.} \\
\end{tabular}
\end{table}


\subsection{Event study}

Figure~\ref{fig:eventstudy} plots monthly DiD coefficients from an event-study specification, interacting month dummies with the high-exposure indicator (omitting February~2020 as the reference period). The absence of systematic pre-trends supports the parallel trends assumption. The coefficients show no divergence at the ChatGPT launch date beyond what is attributable to the Riksbanken hiking cycle.

\begin{figure}[htbp]
\centering
\includegraphics[width=\textwidth]{figA3_event_study.png}
\caption{Event study: monthly DiD coefficients for high vs low genAI exposure occupations (reference: February 2020). Shaded area shows 95\% confidence interval. Dashed lines mark the Riksbanken first rate hike (April 2022) and ChatGPT launch (December 2022).}
\label{fig:eventstudy}
\end{figure}


\subsection{Alternative stock market index}

Figure~\ref{fig:omxspi} replicates the ``scary chart'' using the OMXSPI (OMX Stockholm All-Share Price Index), which covers all companies listed on Nasdaq Stockholm rather than only the 30 largest. The OMXS30 is dominated by banks, industrials, and a few technology firms, raising the concern that the stock market--postings divergence reflects the performance of a narrow set of large caps. The OMXSPI comparison shows that the pattern is virtually identical, confirming that the divergence is not an artefact of index composition.

\begin{figure}[htbp]
\centering
\includegraphics[width=\textwidth]{figA4_omxspi_comparison.png}
\caption{Stock market vs job postings using OMXS30 (left) and OMXSPI All-Share (right). Both panels show the same divergence pattern, confirming it is not driven by the composition of the OMXS30.}
\label{fig:omxspi}
\end{figure}


% ══════════════════════════════════════════════════════════════════════════════
\section{DAIOE exposure distribution}
% ══════════════════════════════════════════════════════════════════════════════

The DAIOE genAI percentile ranking provides a continuous measure of occupational exposure to generative AI capabilities. The distribution across Swedish SSYK~4-digit occupations is approximately uniform by construction (it is a percentile ranking), with quartile boundaries at approximately the 25th, 50th, and 75th percentiles of the occupation distribution.


% ══════════════════════════════════════════════════════════════════════════════
\section{Data documentation}
% ══════════════════════════════════════════════════════════════════════════════

\subsection{Platsbanken}
Historical job advertisement data from the Swedish Public Employment Service (Arbetsf\"ormedlingen), published under CC0 licence. Each record contains: ad identifier, publication date, SSYK~2012 four-digit occupation code, number of vacancies, municipality code, employer name, and source type. Available at \url{https://data.jobtechdev.se/annonser/historiska/}.

\subsection{DAIOE}
The Dynamic AI Occupational Exposure index maps AI benchmark performance to occupational task content. The genAI variant focuses on capabilities relevant to large language models and image generation. Publicly available; see \citet{lodefalk2024daioe}.

\subsection{OMXS30}
Stockholm OMX~30 daily closing prices from Yahoo Finance (ticker: \texttt{\^{}OMX}).

\subsection{Riksbanken policy rate}
Manually compiled from Riksbanken press releases. Key dates verified against \url{https://riksbank.se}.


\end{document}
