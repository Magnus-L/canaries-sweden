% Online Appendix for "Two Economies?"
% This supplementary material does not count toward the 2,000-word limit.

\documentclass[preprint,12pt,authoryear]{elsarticle}

\usepackage[utf8]{inputenc}
\usepackage[T1]{fontenc}
\usepackage{amsmath,amssymb}
\usepackage{graphicx}
\usepackage{booktabs}
\usepackage{hyperref}
\usepackage{natbib}

\graphicspath{{../figures/}}

\begin{document}

\begin{frontmatter}
\title{Online Appendix: Two Economies? Stock Markets, Job Postings, and AI Exposure in Sweden}
\author{Lodefalk, Engberg, Koch, and L\"{o}thman}
\end{frontmatter}

\setcounter{section}{0}
\renewcommand{\thesection}{A\arabic{section}}
\renewcommand{\thetable}{A\arabic{table}}
\renewcommand{\thefigure}{A\arabic{figure}}


% ══════════════════════════════════════════════════════════════════════════════
\section{Sweden vs United States comparison}
% ══════════════════════════════════════════════════════════════════════════════

Figure~\ref{fig:svdus} shows side-by-side comparisons of the stock market versus job postings divergence in the United States and Sweden. The US panel uses the S\&P~500 and Indeed Hiring Lab aggregate job postings index; the Swedish panel uses OMXS30 and Platsbanken microdata. Both countries exhibit a similar qualitative pattern of rising stock markets and falling postings since mid-2022.

\begin{figure}[htbp]
\centering
\includegraphics[width=\textwidth]{figA1_sweden_vs_us.png}
\caption{Stock markets vs job postings: United States (left) and Sweden (right), both indexed to 100 at February 2020.}
\label{fig:svdus}
\end{figure}


% ══════════════════════════════════════════════════════════════════════════════
\section{Individual quartile trends}
% ══════════════════════════════════════════════════════════════════════════════

Figure~\ref{fig:quartiles} shows each AI exposure quartile's posting trajectory individually. All four quartiles exhibit similar cyclical patterns, peaking in mid-2022 and declining through 2023--2024. The parallelism of these trends is consistent with a common macroeconomic driver rather than AI-specific displacement.

\begin{figure}[htbp]
\centering
\includegraphics[width=\textwidth]{figA2_quartile_panels.png}
\caption{Job posting index by genAI exposure quartile (Feb 2020 = 100), individual panels.}
\label{fig:quartiles}
\end{figure}


% ══════════════════════════════════════════════════════════════════════════════
\section{Posting gap: top vs bottom quartile}
% ══════════════════════════════════════════════════════════════════════════════

Figure~\ref{fig:exposure_gap} plots the difference in posting indices between Q4 (highest genAI exposure) and Q1 (lowest) over time. The gap fluctuates around zero throughout the sample period, including after the ChatGPT launch. There is no visible widening of the gap following November 2022, as would be expected if AI were differentially reducing postings in exposed occupations.

\begin{figure}[htbp]
\centering
\includegraphics[width=\textwidth]{fig2_exposure_gap.png}
\caption{Posting index gap between Q4 (highest genAI exposure) and Q1 (lowest), with 3-month moving average and 95\% confidence band. No systematic widening after ChatGPT.}
\label{fig:exposure_gap}
\end{figure}


% ══════════════════════════════════════════════════════════════════════════════
\section{Top and bottom occupations}
% ══════════════════════════════════════════════════════════════════════════════

Table~\ref{tab:topbottom} lists the ten most and ten least genAI-exposed occupations according to the DAIOE index.

\begin{table}[htbp]
\centering
\caption{Most and least genAI-exposed occupations (DAIOE)}
\label{tab:topbottom}
\begin{tabular}{clc}
\hline\hline
SSYK & Occupation & GenAI pctl \\
\hline
\multicolumn{3}{l}{\textit{Most exposed (top 10)}} \\
2641 & Authors and related writers & 99.9 \\
2122 & Statisticians & 99.7 \\
2121 & Mathematicians and actuaries & 99.4 \\
2415 & Economists & 99.2 \\
2512 & Software and systems developers & 98.9 \\
2145 & Chemical engineers & 98.7 \\
2111 & Physicists and astronomers & 98.5 \\
2414 & Securities traders and fund managers & 98.2 \\
2513 & Game and digital media developers & 98.0 \\
2112 & Meteorologists & 97.8 \\
\hline
\multicolumn{3}{l}{\textit{Least exposed (bottom 10)}} \\
9120 & Vehicle, window and related cleaners & 2.2 \\
2653 & Dancers and choreographers & 2.0 \\
8350 & Ships' deck crew and related workers & 1.8 \\
7113 & Concrete workers & 1.5 \\
8341 & Agricultural and forestry machinery operators & 1.3 \\
9310 & Construction labourers & 1.1 \\
8342 & Earth-moving machinery operators & 0.8 \\
8111 & Miners and quarry workers & 0.6 \\
7121 & Roofers & 0.3 \\
3421 & Professional athletes & 0.1 \\
\hline\hline
\end{tabular}
\end{table}


% ══════════════════════════════════════════════════════════════════════════════
\section{Robustness checks}
% ══════════════════════════════════════════════════════════════════════════════

Table~\ref{tab:did} reports the main difference-in-differences estimates discussed in the text (Equation~1 of the main paper).

\begin{table}[htbp]
\centering
\caption{Difference-in-differences: postings by genAI exposure}
\label{tab:did}
\begin{tabular}{lcccc}
\hline\hline
 & (1) & (2) & (3) & (4) \\
 & Monetary & + ChatGPT & + Trends & Group$\times$Time \\
\hline
Post-Riksbank $\times$ High & -0.178*** & -0.127*** & -0.068 & -0.039 \\
 & (0.041) & (0.039) & (0.050) & (0.049) \\
Post-ChatGPT $\times$ High &  & -0.062 & 0.018 & -0.032 \\
 &  & (0.038) & (0.040) & (0.048) \\
Time trend $\times$ High &  &  & -0.003** &  \\
 &  &  & (0.001) &  \\
\hline
Occupation FE & Yes & Yes & Yes & Yes \\
Month FE & Yes & Yes & Yes & \\
Occ.\ group $\times$ month FE & & & & Yes \\
Dep.\ var.\ mean & \multicolumn{4}{c}{3.77} \\
Occupations & \multicolumn{4}{c}{369} \\
Observations & \multicolumn{4}{c}{26,672} \\
\hline\hline
\multicolumn{5}{p{0.95\textwidth}}{\footnotesize \textit{Notes:} Dependent variable: $\ln(\text{postings}_{it})$. High = top quartile of DAIOE genAI percentile ranking. Post-Riksbank $=$ April 2022 onward. Post-ChatGPT $=$ December 2022 onward. Column (4) replaces common month FE with SSYK 1-digit occupation group $\times$ month FE. Standard errors clustered at occupation level in parentheses. $^{***}p<0.01$, $^{**}p<0.05$, $^{*}p<0.10$.} \\
\end{tabular}
\end{table}

Table~\ref{tab:robustness} reports results from alternative specifications: (i)~using the all-apps AI exposure measure instead of genAI; (ii)~using vacancy-weighted postings; (iii)~excluding pandemic months (January--June 2020); (iv)~using tercile classification instead of quartiles; (v)~excluding IT/tech occupations (SSYK~25xx), following \citet{brynjolfsson2025canaries}; (vi)~restricting to a balanced panel of 308 occupations observed in every month; (vii)~using language-modelling task exposure from DAIOE; (viii)~adding quadratic occupation-specific time trends. The ChatGPT coefficient $\hat\beta_2$ is insignificant in six of eight specifications but significant at the 5\% level in the all-apps measure ($\hat\beta_2 = -0.091$, $p = 0.018$) and tercile classification ($\hat\beta_2 = -0.081$, $p = 0.026$). The coefficient is negative in seven of eight specifications.

\begin{table}[htbp]
\centering
\caption{Robustness checks}
\label{tab:robustness}
\begin{tabular}{lcccc}
\hline\hline
Specification & $\hat\beta_1$ (Riksbank) & $\hat\beta_2$ (ChatGPT) & $N$ & Occ. \\
\hline
Baseline (genAI Q4) & -0.1271*** & -0.0615 & 26,672 & 369 \\
 & (0.0388) & (0.0380) & & \\
All-apps measure & -0.0921** & -0.0911** & 26,672 & 369 \\
 & (0.0386) & (0.0385) & & \\
Vacancy-weighted & -0.1038** & -0.0612 & 26,672 & 369 \\
 & (0.0443) & (0.0437) & & \\
Excl. pandemic & -0.1159*** & -0.0617 & 24,479 & 369 \\
 & (0.0356) & (0.0380) & & \\
Terciles & -0.0936*** & -0.0808** & 26,672 & 369 \\
 & (0.0360) & (0.0362) & & \\
Excl. IT/tech & -0.1291*** & -0.0662* & 26,148 & 362 \\
 & (0.0406) & (0.0400) & & \\
Balanced panel & -0.1130*** & -0.0325 & 22,176 & 308 \\
 & (0.0377) & (0.0345) & & \\
Language model & -0.1154*** & -0.0564 & 26,672 & 369 \\
 & (0.0386) & (0.0387) & & \\
Quadratic trends & -0.0884*** & 0.0193 & 26,672 & 369 \\
 & (0.0323) & (0.0402) & & \\
\hline\hline
\multicolumn{5}{p{0.95\textwidth}}{\footnotesize \textit{Notes:} All specifications include occupation and month fixed effects. Standard errors (in parentheses) clustered at occupation level. $^{***}p<0.01$, $^{**}p<0.05$, $^{*}p<0.10$.} \\
\end{tabular}
\end{table}


\subsection{Event study}

Figure~\ref{fig:eventstudy} plots monthly DiD coefficients from an event-study specification, interacting month dummies with the high-exposure indicator (omitting February~2020 as the reference period). The pre-period coefficients fluctuate around zero with no systematic trend, though a joint Wald test rejects the null that all 26 pre-Riksbank coefficients are zero ($\chi^2_{26} = 107.1$, $p < 0.01$). Aggregating to quarterly frequency (Figure~\ref{fig:eventstudy_q}) reduces the number of pre-period coefficients to 9 but the test still rejects ($\chi^2_{8} = 52.8$, $p < 0.01$), indicating genuine differential macro sensitivity across AI exposure groups rather than monthly noise. This motivates specification (4) in the main table, which conditions on occupation group~$\times$~month fixed effects. The post-ChatGPT coefficients show no additional structural break beyond the Riksbank hiking cycle in either specification.

\begin{figure}[htbp]
\centering
\includegraphics[width=\textwidth]{figA3_event_study.png}
\caption{Event study: monthly DiD coefficients for high vs low genAI exposure occupations (reference: February 2020). Shaded area shows 95\% confidence interval. Dashed lines mark the Riksbank's first rate hike (April 2022) and ChatGPT launch (November 2022).}
\label{fig:eventstudy}
\end{figure}


\subsection{Quarterly event study}

Figure~\ref{fig:eventstudy_q} aggregates the monthly event study to quarterly frequency, reducing noise but preserving the key patterns.

\begin{figure}[htbp]
\centering
\includegraphics[width=\textwidth]{figA5_event_study_quarterly.png}
\caption{Quarterly event study: DiD coefficients for high vs low genAI exposure occupations (reference: 2020~Q1). Shaded area shows 95\% confidence interval. Aggregating to quarters reduces monthly noise but pre-period differentials remain jointly significant ($\chi^2_{8} = 52.8$, $p < 0.01$), reflecting differential macro sensitivity.}
\label{fig:eventstudy_q}
\end{figure}


\subsection{Alternative stock market index}

Figure~\ref{fig:omxspi} replicates the ``scary chart'' using the OMXSPI (OMX Stockholm All-Share Price Index), which covers all companies listed on Nasdaq Stockholm rather than only the 30 largest. The OMXS30 is dominated by banks, industrials, and a few technology firms, raising the concern that the stock market--postings divergence reflects the performance of a narrow set of large caps. The OMXSPI comparison shows that the pattern is virtually identical, confirming that the divergence is not an artefact of index composition.

\begin{figure}[htbp]
\centering
\includegraphics[width=\textwidth]{figA4_omxspi_comparison.png}
\caption{Stock market vs job postings using OMXS30 (left) and OMXSPI All-Share (right). Both panels show the same divergence pattern, confirming it is not driven by the composition of the OMXS30.}
\label{fig:omxspi}
\end{figure}


\subsection{Quadratic occupation-specific trends}

Adding quadratic time trends interacted with the high-exposure indicator tests whether non-linear differential dynamics (such as accelerating AI adoption over time) drive the results beyond what linear trends capture. The quadratic term is insignificant ($\hat\delta_2 = -0.00002$, $p = 0.56$), indicating that the linear trend specification (column~3 in the main table) is sufficient. The ChatGPT coefficient remains insignificant ($\hat\beta_2 = 0.019$, $p = 0.63$).


\subsection{Sensitivity to violations of parallel trends}

Figure~\ref{fig:rr} reports a sensitivity analysis following the relative magnitudes framework of \citet{rambachan2023more}. The average post-ChatGPT event-study coefficient for high-exposure occupations is $\hat\theta = -0.169$ (SE = 0.059), significantly negative under exact parallel trends ($\bar{M} = 0$). However, the ``breakdown value'' is $\bar{M} = 0.25$: if post-period violations of parallel trends are as little as 25\% of the maximum pre-period violation, the honest confidence interval includes zero. This confirms that while there is suggestive evidence of a negative AI effect on postings, the finding is fragile, consistent with the imprecision documented in the main regression table.

\begin{figure}[htbp]
\centering
\includegraphics[width=\textwidth]{figA6_rambachan_roth.png}
\caption{Rambachan-Roth sensitivity analysis for the average post-ChatGPT effect on high vs low genAI exposure occupations. The solid line shows the point estimate ($\hat\theta = -0.169$); the shaded area shows the 95\% honest confidence interval as a function of $\bar{M}$ (the maximum ratio of post- to pre-period trend violations). The dotted line marks the breakdown value $\bar{M} = 0.25$.}
\label{fig:rr}
\end{figure}


% ══════════════════════════════════════════════════════════════════════════════
\section{Interest rate sensitivity and AI exposure}
% ══════════════════════════════════════════════════════════════════════════════

A key concern for identification is whether AI-exposed occupations are also more sensitive to monetary policy, which would confound the rate-hike and ChatGPT interactions. To test this, we compute a ``revealed'' monetary policy sensitivity measure: the log change in mean postings between the pre-hike window (January--March 2022) and the post-hike, pre-ChatGPT window (May--November 2022). This window is uncontaminated by AI effects (ChatGPT launched November~30, 2022), so any posting decline is macroeconomic by definition.

Figure~\ref{fig:rate_scatter} plots this revealed rate sensitivity against DAIOE genAI exposure. There is no significant correlation (unweighted $r = 0.035$, $p = 0.507$; posting-weighted $r = -0.020$), indicating that the rate hike affected postings broadly across all AI exposure levels. This mirrors \citet{brynjolfsson2025canaries}, who found a negative correlation between AI exposure and interest rate sensitivity in the US using the \citet{zens2020interest} measure; Sweden shows essentially zero correlation with our revealed-preference approach.

\begin{figure}[htbp]
\centering
\includegraphics[width=\textwidth]{figA_rate_sensitivity_scatter.png}
\caption{AI exposure (DAIOE genAI percentile rank) vs revealed monetary policy sensitivity (log change in postings, May--November vs January--March 2022). Bubble size proportional to total posting volume. No significant correlation: the rate hike affected all AI exposure levels similarly.}
\label{fig:rate_scatter}
\end{figure}


% ══════════════════════════════════════════════════════════════════════════════
\section{Teleworkability robustness (Dingel-Neiman)}
% ══════════════════════════════════════════════════════════════════════════════

AI-exposed occupations overlap substantially with teleworkable occupations \citep{dingel2020many}. To test whether the posting null reflects teleworkability rather than AI, we crosswalk the \citet{dingel2020many} teleworkability classification from SOC~2010 to SSYK~2012 via ISCO-08 and re-run specification~(2) separately for teleworkable and non-teleworkable occupations (split at the median score).

Figure~\ref{fig:telework} reveals heterogeneity. In teleworkable occupations (187~SSYK codes), $\hat\beta_2$ is essentially zero ($-0.005$, $p = 0.917$); the posting decline is entirely captured by the Riksbank interaction ($\hat\beta_1 = -0.138$, $p < 0.01$). In non-teleworkable occupations (182~SSYK codes), $\hat\beta_2 = -0.233$ ($p < 0.01$), indicating a significant additional decline after ChatGPT in high-AI-exposure occupations that cannot adapt through remote work. This suggests that AI displacement of postings, where it occurs, concentrates in occupations where remote work arrangements cannot cushion the impact.

\begin{figure}[htbp]
\centering
\includegraphics[width=\textwidth]{figA_telework_robustness.png}
\caption{ChatGPT coefficient ($\hat\beta_2$) from specification~(2) estimated separately for teleworkable and non-teleworkable occupations (Dingel-Neiman 2020 classification, crosswalked to SSYK~2012 via ISCO-08). Error bars show 95\% confidence intervals. The AI posting effect concentrates in non-teleworkable occupations.}
\label{fig:telework}
\end{figure}


% ══════════════════════════════════════════════════════════════════════════════
\section{Employment by age group and AI exposure}
% ══════════════════════════════════════════════════════════════════════════════

\citet{brynjolfsson2025canaries} find that young US workers in AI-exposed occupations experienced disproportionate employment declines, ``canaries in the coal mine.'' Our posting-based analysis cannot test this age-specific hypothesis directly. As a first check, we use publicly available register data from SCB (Yrkesregistret, table YREG54BAS) providing annual employment counts by SSYK~4-digit occupation and age group for 2020--2024. These annual data show no visible divergence between young workers in high- vs low-AI-exposure occupations after ChatGPT, but three caveats limit their informativeness: (i)~SCB changed the underlying register from RAMS to BAS from reference year 2022, introducing a methodological break at our treatment timing; (ii)~the 2020 pandemic trough as base year inflates all growth rates; (iii)~with only five annual observations, statistical power is limited.

We therefore turn to monthly employer declaration (AGI) register data from SCB's secure research environment (MONA), which provides population-level individual employment records matched to SSYK~4-digit occupations by age and gender. These data reveal a canaries effect: young workers (16--24) in the most AI-exposed occupations experienced steeper employment declines after late 2022, broadly consistent with \citet{brynjolfsson2025canaries}. Swedish college graduation age is high by international standards (median 27--28), so the 16--24 bracket captures pre-graduation and very-early-career workers. The occupation is assigned from the latest available register year (2023); since occupational transitions at the 4-digit level are slow, this introduces minimal measurement error.

\begin{figure}[htbp]
\centering
\includegraphics[width=\textwidth]{figA7_canaries_employment.png}
\caption{Employment by age group and AI exposure, Sweden 2020--2024 (2020 = 100). Data: SCB Yrkesregistret (YREG54BAS). Young = 16--24 years; High AI = top quartile of DAIOE genAI exposure. The dotted line marks ChatGPT launch (November 2022). Note: methodological break (RAMS to BAS) at 2022.}
\label{fig:canaries_emp}
\end{figure}


% ══════════════════════════════════════════════════════════════════════════════
\section{Spotlight occupations: monthly employment by age (MONA)}
% ══════════════════════════════════════════════════════════════════════════════

\citet{brynjolfsson2025canaries} highlight specific ``spotlight'' occupations (software developers and customer service agents) where AI adoption is particularly visible. A Unionen forecast report independently documents declining employment in administration and customer service in Sweden, noting that ``unemployment among white-collar workers has risen faster than among blue-collar workers'' \citep{unionen2025konjunktur}. Using monthly AGI (employer declaration) register data from SCB's secure environment (MONA), we plot employment trajectories by fine age band (22--25, 26--30, 31--34, 35--40, 41--49, 50+) for these occupations, indexed to October 2022 (just before ChatGPT).

The economy-wide employment figure by age group and AI exposure appears as Figure~2 in the main paper. Below we show the occupation-specific spotlight analyses.

\begin{figure}[htbp]
\centering
\includegraphics[width=\textwidth]{figA8a_mona_canaries_softwaredevelopers.png}
\caption{Monthly employment by age group: Software Developers (SSYK~2512). AGI register data, indexed to base month = 100. Vertical lines mark the Riksbank's first rate hike (April 2022) and ChatGPT launch (November 2022). The youngest cohort (22--25) shows the steepest decline, consistent with the economy-wide canaries effect in Figure~2 of the main paper.}
\label{fig:spotlight_software}
\end{figure}

\begin{figure}[htbp]
\centering
\includegraphics[width=\textwidth]{figA8b_mona_canaries_customerservice.png}
\caption{Monthly employment by age group: Customer Service Agents (SSYK~4221, 4222, 5230). AGI register data, indexed to base month = 100. Vertical lines mark the Riksbank's first rate hike (April 2022) and ChatGPT launch (November 2022). The pattern is noisier than for software developers, potentially reflecting smaller cell sizes and more heterogeneous task content.}
\label{fig:spotlight_customer}
\end{figure}


% ══════════════════════════════════════════════════════════════════════════════
\section{Employer-level difference-in-differences}
\label{sec:employer_did}
% ══════════════════════════════════════════════════════════════════════════════

The descriptive evidence in Figure~2 of the main paper and the triple-difference regression in the preceding sections establish that young workers in high-AI-exposure occupations experienced disproportionate employment declines. However, this occupation-level analysis cannot rule out that unobserved employer-level shocks --- for instance, differential retrenchment by large employers concentrated in AI-exposed occupations --- drive the pattern. To tighten identification, we estimate an employer-level difference-in-differences design closely following \citet{brynjolfsson2025canaries}.

\paragraph{Estimating equation.}
For each age group $a \in \{16\text{--}24,\; 25\text{--}30,\; 31\text{--}40,\; 41\text{--}49,\; 50+\}$, we estimate:
\begin{equation}\label{eq:employer_did}
\ln(\text{emp}^{a}_{fqt}) \;=\; \alpha_{fq} \;+\; \beta_{ft} \;+\; \gamma_1 \cdot \text{PostRB}_t \cdot \text{HighQ4}_q \;+\; \gamma_2 \cdot \text{PostGPT}_t \cdot \text{HighQ4}_q \;+\; \varepsilon_{fqt}
\end{equation}
where $f$ indexes employers, $q \in \{1,2,3,4\}$ indexes DAIOE genAI exposure quartiles, and $t$ indexes year-months. $\alpha_{fq}$ are employer $\times$ quartile fixed effects and $\beta_{ft}$ are employer $\times$ month fixed effects. The dependent variable is the log count of workers of age group $a$ employed by firm $f$ in occupations belonging to quartile $q$ at time $t$. $\text{PostRB}_t = 1$ if $t \geq$ April 2022; $\text{PostGPT}_t = 1$ if $t \geq$ December 2022; $\text{HighQ4}_q = 1$ if the occupation belongs to the top quartile of genAI exposure. Standard errors are clustered at the employer $\times$ quartile level.

\paragraph{Identification.}
The employer $\times$ month fixed effects $\beta_{ft}$ are the key identification device. They absorb \emph{all} time-varying employer-level shocks --- including firm-specific responses to monetary tightening, demand shocks, management changes, or strategic decisions --- leaving only \emph{within-firm, within-month} variation across AI exposure quartiles. The identifying assumption is that, absent AI, employers would have adjusted employment in high- and low-exposure occupations proportionally. This is substantially weaker than the parallel trends assumption required by the occupation-level analysis, which requires that aggregate shocks affect all occupations equally.

\paragraph{Canaries test.}
The coefficient $\hat\gamma_2$ in Equation~\eqref{eq:employer_did} measures the differential employment change in high-AI-exposure occupations (relative to lower quartiles, within the same employer, in the same month) after ChatGPT. By estimating the equation separately for each age group, we test whether this within-employer recomposition is age-graded. If $\hat\gamma_2$ is significantly negative for the youngest workers (16--24) but not for older groups, this constitutes evidence of a canaries effect: employers differentially reduce young hires in AI-exposed roles after the advent of generative AI.

\paragraph{Event study.}
To validate the parallel trends assumption and trace the dynamics of any effect, we also estimate an event-study specification:
\begin{equation}\label{eq:employer_es}
\ln(\text{emp}^{a}_{fqt}) \;=\; \alpha_{fq} \;+\; \beta_{ft} \;+\; \sum_{s \neq s_0} \delta_s \cdot \mathbf{1}[t \in s] \cdot \text{HighQ4}_q \;+\; \varepsilon_{fqt}
\end{equation}
where $s$ indexes half-year periods and $s_0 = $ 2022H1 is the reference period (the last half-year before the Riksbank hike). Pre-treatment coefficients $\hat\delta_s$ for $s < s_0$ should be statistically indistinguishable from zero if the parallel trends assumption holds within employers.

\paragraph{Sample construction.}
We use the same monthly AGI register data as the descriptive analysis, retaining employers with at least one worker observed in both high (Q4) and low (Q1--Q3) exposure occupations. This restriction ensures that identification comes from within-employer comparisons; single-quartile employers contribute no variation. Each employer-quartile-month cell records the count of workers in the relevant age group, and cells with zero employment are included to capture extensive-margin adjustment.

\paragraph{Estimation.}
The two-way high-dimensional fixed effects are absorbed using the method of alternating projections (\texttt{PanelOLS} in Python; \texttt{fixest} in R). With potentially millions of employer $\times$ month fixed effects, memory constraints may bind; we implement a three-level estimation fallback: (i) direct absorption via \texttt{linearmodels}/\texttt{fixest}, (ii) manual within-transformation (demeaning), (iii) occupation-level aggregation if employer-level estimation is infeasible.

[Results to be added upon completion of the MONA analysis.]


% ══════════════════════════════════════════════════════════════════════════════
\section{Riksbank policy rate timeline}
% ══════════════════════════════════════════════════════════════════════════════

Figure~\ref{fig:riksbank_rate} shows the Riksbank's policy rate from 2020 through early 2026. The tightening cycle ran from April 2022 (first hike, 0\% to 0.25\%) through September 2023 (peak at 4.00\%), followed by gradual easing beginning in May 2024. The 7-month gap between the first hike and ChatGPT's launch (November 2022) is the identification window that separates monetary tightening from AI exposure in our design.

\begin{figure}[htbp]
\centering
\includegraphics[width=\textwidth]{figA10_riksbank_rate.png}
\caption{Riksbank policy rate, 2020--2026. The tightening cycle ran from April 2022 to September 2023 (0\% to 4.00\%), with easing from May 2024. The 7-month gap between the first hike and ChatGPT's launch is marked.}
\label{fig:riksbank_rate}
\end{figure}


% ══════════════════════════════════════════════════════════════════════════════
\section{Half-year posting event study}
% ══════════════════════════════════════════════════════════════════════════════

Figure~\ref{fig:halfyear_es} aggregates the event study to half-year periods, with 2022H1 (pre-Riksbank hike) as the reference. Pre-period coefficients are positive and significant, indicating that high-AI-exposure occupations had relatively more postings before the treatment period. After the reference period, coefficients drift negative: the 2023H1 coefficient is marginally significant ($-0.066$, $p = 0.08$), while 2025H1 and 2025H2 reach marginal significance ($-0.094$ and $-0.099$, both $p < 0.10$). The 2026H1 estimate ($-0.412$, $p < 0.01$) is based on only two months and should be interpreted cautiously. The gradual negative drift, rather than a sharp break at the ChatGPT launch (2022H2), is consistent with the main finding that the posting decline is driven by macroeconomic factors with at most a small additional AI component.

\begin{figure}[htbp]
\centering
\includegraphics[width=\textwidth]{figA11_halfyear_event_study.png}
\caption{Half-year event study: coefficients on High $\times$ half-year period interactions (reference: 2022H1). Occupation and month fixed effects; standard errors clustered by occupation. Shaded area: 95\% confidence interval. The gradual negative drift after 2022 is consistent with broad-based macro effects rather than a discrete AI shock.}
\label{fig:halfyear_es}
\end{figure}


% ══════════════════════════════════════════════════════════════════════════════
\section{DAIOE exposure distribution}
% ══════════════════════════════════════════════════════════════════════════════

The DAIOE genAI percentile ranking provides a continuous measure of occupational exposure to generative AI capabilities. The distribution across Swedish SSYK~4-digit occupations is approximately uniform by construction (it is a percentile ranking), with quartile boundaries at approximately the 25th, 50th, and 75th percentiles of the occupation distribution.


% ══════════════════════════════════════════════════════════════════════════════
\section{Data documentation}
% ══════════════════════════════════════════════════════════════════════════════

\subsection{Platsbanken}
Job advertisement data from the Swedish Public Employment Service (Arbetsf\"ormedlingen), published under CC0 licence. Each record contains: ad identifier, publication date, SSYK~2012 four-digit occupation code, number of vacancies, municipality code, employer name, and source type. For 2020--2025 we use the historical bulk archive (\url{https://data.jobtechdev.se/annonser/historiska/}); for January--February 2026 we supplement with the real-time JobStream API (\url{https://jobstream.api.jobtechdev.se}), which provides the same fields for currently published ads. The two sources are deduplicated on ad identifier before aggregation.

\subsection{DAIOE}
The Dynamic AI Occupational Exposure index maps AI benchmark performance to occupational task content. The genAI variant focuses on capabilities relevant to large language models and image generation. Publicly available; see \citet{engberg2024daioe}.

\subsection{OMXS30}
Stockholm OMX~30 daily closing prices from Yahoo Finance (ticker: \texttt{\^{}OMX}).

\subsection{Riksbank policy rate}
Manually compiled from Riksbank press releases. Key dates verified against \url{https://riksbank.se}. Figure~\ref{fig:riksbank_rate} plots the full tightening cycle.

\subsection{Administrative employment data (SCB/MONA)}

The employment analysis uses monthly employer declaration (AGI, \emph{arbetsgivardeklaration}) register data from Statistics Sweden (SCB). These data are drawn from the ORU-MICRO-AI employer--employee database of \"Orebro University, which links several administrative registers. The variables used in this paper are: encrypted person identifier, SSYK~2012 four-digit occupation code (from the 2023 register year), age, gender, employer identifier, and monthly employment status. The data cover the population of employed individuals in Sweden, with monthly frequency from 2019 onward.

The administrative micro-level data are from several registers of Statistics Sweden and are contained in the ORU-MICRO-AI database, obtained after approval from the Swedish Ethical Review Authority (approvals 2021-05040, 2022-03330-02, 2024-01714-0, 2025-04205-02). Researchers and their assistants may access these and similar Swedish micro-level data, subject to approval by SCB. Access is provided through the secure internet-based Micro-data Online Access (MONA) system; microdata never leave SCB's servers. Foreign-based researchers may visit a Swedish institution with access or cooperate with researchers in Sweden. Information on the application process is available at \url{https://www.scb.se/en/services/ordering-data-and-statistics/ordering-microdata/} or by contacting SCB at \texttt{mona@scb.se}. Researchers wishing to access our specific data for replication purposes should contact the corresponding author for guidance regarding project approval from the Swedish Ethical Review Authority and SCB.

\subsection{Danish administrative employment data (DST)}

The supplementary Danish analysis uses monthly employment register data from Statistics Denmark (Danmarks Statistik, DST). Variables used are: encrypted person identifier, DISCO~4-digit occupation code (the Danish adaptation of ISCO-08, closely aligned with SSYK~2012 at the 4-digit level), birth year, and monthly employment status. Researchers may apply for access to Danish register data by submitting a research proposal to Statistics Denmark. The application must include a detailed project description and documentation of institutional affiliation. Access is subject to Danish data protection regulations, including the EU General Data Protection Regulation (GDPR). Information on the application process is available at \url{https://www.dst.dk/en/TilSalg/Forskningsservice}. Researchers wishing to access the specific data used for replication purposes should contact the corresponding author.

All replication code for both public and restricted data is available at \url{https://github.com/Magnus-L/canaries-sweden}. The authors commit to preserving data access and providing reasonable assistance to replication requests for at least five years following publication.


\bibliographystyle{elsarticle-harv}
\bibliography{references}

\end{document}
