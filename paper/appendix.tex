% Online Appendix for "Two Economies?"
% This supplementary material does not count toward the 2,000-word limit.

\documentclass[preprint,12pt,authoryear]{elsarticle}

\usepackage[utf8]{inputenc}
\usepackage[T1]{fontenc}
\usepackage{amsmath,amssymb}
\usepackage{graphicx}
\usepackage{booktabs}
\usepackage{hyperref}
\usepackage{natbib}

\graphicspath{{../figures/}}

\begin{document}

\begin{frontmatter}
\title{Online Appendix: Two Economies? Stock Markets, Job Postings, and AI Exposure in Sweden}
\author{Lodefalk, Engberg, Koch, and L\"{o}thman}
\end{frontmatter}

\setcounter{section}{0}
\renewcommand{\thesection}{A\arabic{section}}
\renewcommand{\thetable}{A\arabic{table}}
\renewcommand{\thefigure}{A\arabic{figure}}


% ══════════════════════════════════════════════════════════════════════════════
\section{Sweden vs United States comparison}
% ══════════════════════════════════════════════════════════════════════════════

Figure~\ref{fig:svdus} shows side-by-side comparisons of the stock market versus job postings divergence in the United States and Sweden. The US panel uses the S\&P~500 and Indeed Hiring Lab aggregate job postings index; the Swedish panel uses OMXS30 and Platsbanken microdata. Both countries exhibit a similar qualitative pattern of rising stock markets and falling postings since mid-2022.

\begin{figure}[htbp]
\centering
\includegraphics[width=\textwidth]{figA1_sweden_vs_us.png}
\caption{Stock markets vs job postings: United States (left) and Sweden (right), both indexed to 100 at February 2020.}
\label{fig:svdus}
\end{figure}


% ══════════════════════════════════════════════════════════════════════════════
\section{Individual quartile trends}
% ══════════════════════════════════════════════════════════════════════════════

Figure~\ref{fig:quartiles} shows each AI exposure quartile's posting trajectory individually. All four quartiles exhibit similar cyclical patterns, peaking in mid-2022 and declining through 2023--2024. The parallelism of these trends is consistent with a common macroeconomic driver rather than AI-specific displacement.

\begin{figure}[htbp]
\centering
\includegraphics[width=\textwidth]{figA2_quartile_panels.png}
\caption{Job posting index by genAI exposure quartile (Feb 2020 = 100), individual panels.}
\label{fig:quartiles}
\end{figure}


% ══════════════════════════════════════════════════════════════════════════════
\section{Top and bottom occupations}
% ══════════════════════════════════════════════════════════════════════════════

Table~\ref{tab:topbottom} lists the ten most and ten least genAI-exposed occupations according to the DAIOE index.

\begin{table}[htbp]
\centering
\caption{Most and least genAI-exposed occupations (DAIOE)}
\label{tab:topbottom}
\begin{tabular}{clc}
\hline\hline
SSYK & Occupation & GenAI pctl \\
\hline
\multicolumn{3}{l}{\textit{Most exposed (top 10)}} \\
2641 & Authors and related writers & 99.9 \\
2122 & Statisticians & 99.7 \\
2121 & Mathematicians and actuaries & 99.4 \\
2415 & Economists & 99.2 \\
2512 & Software and systems developers & 98.9 \\
2145 & Chemical engineers & 98.7 \\
2111 & Physicists and astronomers & 98.5 \\
2414 & Securities traders and fund managers & 98.2 \\
2513 & Game and digital media developers & 98.0 \\
2112 & Meteorologists & 97.8 \\
\hline
\multicolumn{3}{l}{\textit{Least exposed (bottom 10)}} \\
9120 & Vehicle, window and related cleaners & 2.2 \\
2653 & Dancers and choreographers & 2.0 \\
8350 & Ships' deck crew and related workers & 1.8 \\
7113 & Concrete workers & 1.5 \\
8341 & Agricultural and forestry machinery operators & 1.3 \\
9310 & Construction labourers & 1.1 \\
8342 & Earth-moving machinery operators & 0.8 \\
8111 & Miners and quarry workers & 0.6 \\
7121 & Roofers & 0.3 \\
3421 & Professional athletes & 0.1 \\
\hline\hline
\end{tabular}
\end{table}


% ══════════════════════════════════════════════════════════════════════════════
\section{Robustness checks}
% ══════════════════════════════════════════════════════════════════════════════

Table~\ref{tab:robustness} reports results from alternative specifications: (i)~using the all-apps AI exposure measure instead of genAI; (ii)~using vacancy-weighted postings; (iii)~excluding pandemic months (January--June 2020); (iv)~using tercile classification instead of quartiles; (v)~excluding IT/tech occupations (SSYK~25xx), following \citet{brynjolfsson2025canaries}; (vi)~restricting to a balanced panel of 308 occupations observed in every month; (vii)~using language-modelling task exposure from DAIOE; (viii)~adding quadratic occupation-specific time trends. The ChatGPT coefficient $\hat\beta_2$ is insignificant in six of eight specifications but significant at the 5\% level in the all-apps measure ($\hat\beta_2 = -0.091$, $p = 0.018$) and tercile classification ($\hat\beta_2 = -0.081$, $p = 0.026$). The coefficient is negative in seven of eight specifications.

\begin{table}[htbp]
\centering
\caption{Robustness checks}
\label{tab:robustness}
\begin{tabular}{lcccc}
\hline\hline
Specification & $\hat\beta_1$ (Riksbank) & $\hat\beta_2$ (ChatGPT) & $N$ & Occ. \\
\hline
Baseline (genAI Q4) & -0.1271*** & -0.0615 & 26,672 & 369 \\
 & (0.0388) & (0.0380) & & \\
All-apps measure & -0.0921** & -0.0911** & 26,672 & 369 \\
 & (0.0386) & (0.0385) & & \\
Vacancy-weighted & -0.1038** & -0.0612 & 26,672 & 369 \\
 & (0.0443) & (0.0437) & & \\
Excl. pandemic & -0.1159*** & -0.0617 & 24,479 & 369 \\
 & (0.0356) & (0.0380) & & \\
Terciles & -0.0936*** & -0.0808** & 26,672 & 369 \\
 & (0.0360) & (0.0362) & & \\
Excl. IT/tech & -0.1291*** & -0.0662* & 26,148 & 362 \\
 & (0.0406) & (0.0400) & & \\
Balanced panel & -0.1130*** & -0.0325 & 22,176 & 308 \\
 & (0.0377) & (0.0345) & & \\
Language model & -0.1154*** & -0.0564 & 26,672 & 369 \\
 & (0.0386) & (0.0387) & & \\
Quadratic trends & -0.0884*** & 0.0193 & 26,672 & 369 \\
 & (0.0323) & (0.0402) & & \\
\hline\hline
\multicolumn{5}{p{0.95\textwidth}}{\footnotesize \textit{Notes:} All specifications include occupation and month fixed effects. Standard errors (in parentheses) clustered at occupation level. $^{***}p<0.01$, $^{**}p<0.05$, $^{*}p<0.10$.} \\
\end{tabular}
\end{table}


\subsection{Event study}

Figure~\ref{fig:eventstudy} plots monthly DiD coefficients from an event-study specification, interacting month dummies with the high-exposure indicator (omitting February~2020 as the reference period). The pre-period coefficients fluctuate around zero with no systematic trend, though a joint Wald test rejects the null that all 26 pre-Riksbank coefficients are zero ($\chi^2_{26} = 107.1$, $p < 0.01$). Aggregating to quarterly frequency (Figure~\ref{fig:eventstudy_q}) reduces the number of pre-period coefficients to 9 but the test still rejects ($\chi^2_{8} = 52.8$, $p < 0.01$), indicating genuine differential macro sensitivity across AI exposure groups rather than monthly noise. This motivates specification (4) in the main table, which conditions on occupation group~$\times$~month fixed effects. The post-ChatGPT coefficients show no additional structural break beyond the Riksbanken hiking cycle in either specification.

\begin{figure}[htbp]
\centering
\includegraphics[width=\textwidth]{figA3_event_study.png}
\caption{Event study: monthly DiD coefficients for high vs low genAI exposure occupations (reference: February 2020). Shaded area shows 95\% confidence interval. Dashed lines mark the Riksbanken first rate hike (April 2022) and ChatGPT launch (December 2022).}
\label{fig:eventstudy}
\end{figure}


\subsection{Quarterly event study}

Figure~\ref{fig:eventstudy_q} aggregates the monthly event study to quarterly frequency, reducing noise but preserving the key patterns.

\begin{figure}[htbp]
\centering
\includegraphics[width=\textwidth]{figA5_event_study_quarterly.png}
\caption{Quarterly event study: DiD coefficients for high vs low genAI exposure occupations (reference: 2020~Q1). Shaded area shows 95\% confidence interval. Aggregating to quarters reduces monthly noise but pre-period differentials remain jointly significant ($\chi^2_{8} = 52.8$, $p < 0.01$), reflecting differential macro sensitivity.}
\label{fig:eventstudy_q}
\end{figure}


\subsection{Alternative stock market index}

Figure~\ref{fig:omxspi} replicates the ``scary chart'' using the OMXSPI (OMX Stockholm All-Share Price Index), which covers all companies listed on Nasdaq Stockholm rather than only the 30 largest. The OMXS30 is dominated by banks, industrials, and a few technology firms, raising the concern that the stock market--postings divergence reflects the performance of a narrow set of large caps. The OMXSPI comparison shows that the pattern is virtually identical, confirming that the divergence is not an artefact of index composition.

\begin{figure}[htbp]
\centering
\includegraphics[width=\textwidth]{figA4_omxspi_comparison.png}
\caption{Stock market vs job postings using OMXS30 (left) and OMXSPI All-Share (right). Both panels show the same divergence pattern, confirming it is not driven by the composition of the OMXS30.}
\label{fig:omxspi}
\end{figure}


\subsection{Quadratic occupation-specific trends}

Adding quadratic time trends interacted with the high-exposure indicator tests whether non-linear differential dynamics --- such as accelerating AI adoption over time --- drive the results beyond what linear trends capture. The quadratic term is insignificant ($\hat\delta_2 = -0.00002$, $p = 0.56$), indicating that the linear trend specification (column~3 in the main table) is sufficient. The ChatGPT coefficient remains insignificant ($\hat\beta_2 = 0.019$, $p = 0.63$).


\subsection{Sensitivity to violations of parallel trends}

Figure~\ref{fig:rr} reports a sensitivity analysis following the relative magnitudes framework of \citet{rambachan2023more}. The average post-ChatGPT event-study coefficient for high-exposure occupations is $\hat\theta = -0.169$ (SE = 0.059), significantly negative under exact parallel trends ($\bar{M} = 0$). However, the ``breakdown value'' is $\bar{M} = 0.25$: if post-period violations of parallel trends are as little as 25\% of the maximum pre-period violation, the honest confidence interval includes zero. This confirms that while there is suggestive evidence of a negative AI effect on postings, the finding is fragile --- consistent with the imprecision documented in the main regression table.

\begin{figure}[htbp]
\centering
\includegraphics[width=\textwidth]{figA6_rambachan_roth.png}
\caption{Rambachan-Roth sensitivity analysis for the average post-ChatGPT effect on high vs low genAI exposure occupations. The solid line shows the point estimate ($\hat\theta = -0.169$); the shaded area shows the 95\% honest confidence interval as a function of $\bar{M}$ (the maximum ratio of post- to pre-period trend violations). The dotted line marks the breakdown value $\bar{M} = 0.25$.}
\label{fig:rr}
\end{figure}


% ══════════════════════════════════════════════════════════════════════════════
\section{Employment by age group and AI exposure}
% ══════════════════════════════════════════════════════════════════════════════

\citet{brynjolfsson2025canaries} find that young US workers in AI-exposed occupations experienced disproportionate employment declines --- ``canaries in the coal mine.'' Our posting-based analysis cannot test this age-specific hypothesis. As a supplementary check, we use publicly available register data from SCB (Yrkesregistret, table YREG54BAS) providing annual employment counts by SSYK~4-digit occupation and age group for 2020--2024.

Figure~\ref{fig:canaries_emp} shows employment indexed to 2020, for four groups defined by age (16--24 vs 25+) and AI exposure (top quartile vs rest). All groups recover strongly from the 2020 pandemic trough, with youth employment growing fastest. There is no visible divergence between young workers in high- vs low-AI-exposure occupations after ChatGPT. A triple-difference regression (occupation-age entity and year fixed effects, clustered at entity level) yields an insignificant interaction: $\hat\beta_{\text{Post} \times \text{Young} \times \text{High}} = 0.038$ ($p > 0.10$), providing no evidence of a canaries effect.

Three caveats apply: (i)~SCB changed the underlying register from RAMS to BAS from reference year 2022, introducing a methodological break at our treatment timing; (ii)~the 2020 pandemic trough as base year inflates all growth rates; (iii)~with only five annual observations, statistical power is limited. Monthly employer declaration (AGI) data, available in Sweden's secure research environment (MONA), would permit a more granular test of this hypothesis.

\begin{figure}[htbp]
\centering
\includegraphics[width=\textwidth]{figA7_canaries_employment.png}
\caption{Employment by age group and AI exposure, Sweden 2020--2024 (2020 = 100). Data: SCB Yrkesregistret (YREG54BAS). Young = 16--24 years; High AI = top quartile of DAIOE genAI exposure. The dotted line marks ChatGPT launch (November 2022). Note: methodological break (RAMS to BAS) at 2022.}
\label{fig:canaries_emp}
\end{figure}


% ══════════════════════════════════════════════════════════════════════════════
\section{DAIOE exposure distribution}
% ══════════════════════════════════════════════════════════════════════════════

The DAIOE genAI percentile ranking provides a continuous measure of occupational exposure to generative AI capabilities. The distribution across Swedish SSYK~4-digit occupations is approximately uniform by construction (it is a percentile ranking), with quartile boundaries at approximately the 25th, 50th, and 75th percentiles of the occupation distribution.


% ══════════════════════════════════════════════════════════════════════════════
\section{Data documentation}
% ══════════════════════════════════════════════════════════════════════════════

\subsection{Platsbanken}
Historical job advertisement data from the Swedish Public Employment Service (Arbetsf\"ormedlingen), published under CC0 licence. Each record contains: ad identifier, publication date, SSYK~2012 four-digit occupation code, number of vacancies, municipality code, employer name, and source type. Available at \url{https://data.jobtechdev.se/annonser/historiska/}.

\subsection{DAIOE}
The Dynamic AI Occupational Exposure index maps AI benchmark performance to occupational task content. The genAI variant focuses on capabilities relevant to large language models and image generation. Publicly available; see \citet{lodefalk2024daioe}.

\subsection{OMXS30}
Stockholm OMX~30 daily closing prices from Yahoo Finance (ticker: \texttt{\^{}OMX}).

\subsection{Riksbanken policy rate}
Manually compiled from Riksbanken press releases. Key dates verified against \url{https://riksbank.se}.


\bibliographystyle{elsarticle-harv}
\bibliography{references}

\end{document}
